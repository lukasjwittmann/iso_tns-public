Quantum many-body systems give rise to rich emergent phenomena in their ground states and elementary excitations. Extracting theoretical predictions for experimentally measurable quantities---such as the dynamical structure factor---requires numerical methods that overcome the exponential growth of the Hilbert space dimension with system size. On finite and infinite chains, ground states of local, gapped Hamiltonians can be faithfully approximated within the manifold of matrix product states (MPS). Algorithms for finding these representations are well established in form of the density matrix renormalization group (DMRG) and variational uniform matrix product states (VUMPS). For a ground state in the smooth manifold, the tangent space is spanned by local tensor perturbations and is therefore well suited to variationally optimize quasiparticle excitations therein. All these 1D methods benefit from an isometric canonical form that can be reached exactly within the internal gauge degrees of freedom. 
In two dimensions, projected entangled pair states (PEPS) provide the natural generalization of MPS. Because of closed loops in the virtual network, an analogous canonical form can only be imposed on a strict variational subset. Isometric projected entangled pair states (isoPEPS) enforce an orthogonality column, which enables a generalization of the successful 1D DMRG algorithm to 2D, dubbed $\text{DMRG}^2$. The main challenges concern shifting the orthogonality column with minimal error and efficiently compressing the boundaries of the energy expectation value. 
The first part of this thesis is devoted to reproducing the ansätze and algorithms in 1D. Building on this foundation, we study isoPEPS on the diagonal square lattice in the second part. In particular, we newly propose \textbf{(i) a bulk-weighted boundary compression scheme} that explicitly exploits the isometric form to improve the performance of $\text{DMRG}^2$, and \textbf{(ii) an ansatz for variational quasiparticle excitations}. Our Python implementations of all algorithms are openly available in a git repository \cite{wittmann2025iso}. We perform benchmarks on the transverse field Ising model and find that the quasiparticle ansatz faithfully captures low-energy excitations.