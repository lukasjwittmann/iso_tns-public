Quantum many-body systems exhibit rich emergent phenomena. When sufficiently cooled down, most spin systems spontaneously break a symmetry and show magnetic long-range order. However, there are more exotic and less understood strongly correlated phases of matter, such as quantum spin liquids (QSL) in two dimensions. Features that qualify a system as such are topological ground state degeneracy, long-range entanglement and the emergence of fractionalized quasiparticle excitations. Numerical methods are indispensable for making quantitative theoretical predictions about observables---such as the dynamical structure factor---that are experimentally measurable in candidate materials. \cite{knolle2019field}

The central challenge in numerical approaches is the exponential growth of the Hilbert space dimension with system size. Exact diagonalization (ED) methods, which directly tackle the energy eigenvalue problem, are therefore limited to a few tens of spins, even when employing iterative Lanczos algorithms and exploiting symmetries of the Hamiltonian. Quantum Monte Carlo (QMC) methods do not operate on the high dimensional Hamiltonian matrices and state vectors explicitly, but stochastically sample over configuration spaces. While being extremely efficient for bipartite systems, for geometrically frustrated spin systems---prime QSL candidates---QMC methods are plagued by the sign problem: statistical errors get out of control due to fluctuating signs of the configuration weights. \cite{sandvik2010computational}

Approximating the wavefunction with a network of low-dimensional tensors and not suffering from the sign problem, tensor network state methods open up a promising third path. The density matrix renormalization group (DMRG) algorithm \cite{white1992density}, later reformulated as a variational optimization over matrix product states (MPS) \cite{schollwock2011density}, is the state-of-the-art method for finding ground states in one dimension. Its success can be explained with the area law of entanglement \cite{hastings2007area} for ground states of local, gapped 1D Hamiltonians and the ability of MPS to represent such states faithfully \cite{verstraete2006matrix}. In the thermodynamic limit, DMRG can be applied in its infinite version, which successively grows the unit cell and recovers translation invariance asymptotically. Alternatively, the variational uniform matrix product states (VUMPS) algorithm \cite{zauner2018variational} completely updates uniform MPS (uMPS) with each iteration and therefore keeps translation invariance at any time. Identifying the set of matrix product states as a smooth complex manifold \cite{haegeman2014geometry} provided the geometric foundation for tangent space based approaches \cite{haegeman2013post}. The time-dependent variational principle (TDVP) projects the right hand side of the Schrödinger equation onto the tangent space, ensuring that the time evolution never leaves the manifold. It was formulated both for the infinite uniform \cite{haegeman2011time} and finite \cite{haegeman2016unifying} MPS. Moreover, since the tangent space results from infinitesimal deviations from the original ground state tensors, it is very natural to think of it as the space in which elementary excitations live. For uMPS, a plane wave superposition with definite momentum can be formed from the states in which the tensor is replaced on a single site \cite{haegeman2012variational, vanderstraeten2019tangent}. For finite MPS with open boundary conditions, the local perturbations are superposed without explicit momentum coefficient \cite{van2021efficient}. In both cases, the perturbation parameters have to be optimized energetically. As for the ground state and any other effective eigenvalue problem, isometric conditions improve the numerical stability. For matrix product states with open boundary conditions, such a canonical form can be achieved exactly within the internal gauge degrees of freedom. This is not the case when generalizing MPS to two dimensions, which brings us to projected entangled pair states (PEPS) \cite{verstraete2004renormalization}. Although PEPS offer great variational power for representing area law states, the presence of closed loops in the virtual legs prevents the existence of an analogous canonical form and thus leads to generalized eigenvalue problems. This was the motivation for Zaletel and Pollmann \cite{zaletel2020isometric} to further restrict the variational ansatz to isometric projected entangled pair states (isoPEPS). Although forming a strict PEPS subset, isoPEPS are conjectured to capture all gapped phases with gappable edges \cite{soejima2020isometric}. The imposed orthogonality column enables an algorithm dubbed $\text{DMRG}^{2}$ \cite{lin2022efficient}, the generalization of the successful 1D DMRG algorithm to 2D. The main challenges here concern shifting the orthogonality column as error-free as possible, and finding an efficient way to compress the boundary terms in the energy expectation value. To our knowledge, a quasiparticle excitation ansatz has so far only been studied for infinite PEPS \cite{vanderstraeten2015excitations, vanderstraeten2019simulating}, but not yet for isoPEPS. In this thesis, we work on a diagonal square lattice, for which an isometric form was recently introduced in \cite{sappler2025diagonal}. In particular, we develop two new methods:
\begin{enumerate}
	\item[1)] In section \ref{sec:bc} we propose a new bulk-weighted boundary compression scheme that explicitly exploits the isometric structure of isoPEPS. Compared to the standard variational compression, the energy expectation value converges much faster with boundary bond dimension. In accordance with this, $\text{DMRG}^2$ achieves significantly lower energy errors, which we demonstrate in section \ref{sec:dmrg2}.
	\item[2)] In section \ref{sec:vqpe2} we introduce an isoPEPS quasiparticle excitation ansatz. It parametrizes the tangent space vector such that all desired orthogonality properties are fulfilled. We optimize it (i) from overlap with exact wavefunctions or MPS, and (ii) by diagonalization of the effective Hamiltonian. 
\end{enumerate}
Along the way, we cover the following: In chapter \ref{ch:tfi_model} we present the transverse field Ising (TFI) model, which is the paradigmatic model of quantum many-body spin systems and serves as a benchmark for all algorithms in this thesis. After having equipped ourselves with the basic tensor network tools and notations in chapter \ref{ch:tensor_networks}, we deal with the manifold of uMPS in chapter \ref{ch:umps}. We reproduce the established algorithms for variationally finding the ground state and quasiparticle excitations on top. We follow the same agenda for finite MPS in chapter \ref{ch:mps}. Finally, in chapter \ref{ch:iso_peps}, we set the stage for isoPEPS on the diagonal square lattice in section \ref{sec:canonical_form_yb}, followed by a presentation of our newly developed methods 1) and 2). We close with conclusion and outlook in chapter \ref{ch:conclusion}. All described algorithms are implemented in open-source Python code \cite{wittmann2025iso}.